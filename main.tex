\documentclass{scrartcl}
\usepackage[utf8]{inputenc}
\usepackage{multicol}

\title{IF: Idea Finder}
\subtitle{searching if idea exists or not}
\author{
  \begin{tabular}{cc}
    \shortstack{Kyeong-Tae Koo \\
dept. of Information Systems\\ 2014005296}
 &
   \shortstack{Oh-Hyeon Kwon\\
dept. of Information Systems \\ 2014005323} \\ 
    \shortstack{Jihye Lee \\
dept. of Information Systems\\ 2015005596}
 &
   \shortstack{Juyoun Baek\\
dept. of Tourisms\\ 2015019116}
\end{tabular}
    }
    

\begin{document}
\maketitle    
\begin{abstract}
 \textbf{\textit{Abstract— We would like to suggest, IF, a website to find out if an idea exists. IF contains not only the presence of the subject we are looking for, but also the ability to score how novel the topic is. As far as we know, IF is the first website to evaluate your ideas conveniently and simply and is different from previous search websites.}}
\end{abstract}

\section{INTRODUCTION}
When planning product or software, finding a new idea is the one of the most difficult processes. People should search on many web sites to judge whether the idea exists or not. Also, there were differences in the search result of each site. To solve the inconvenience, building a web site that searches and judges the idea could be helpful. Therefore, we add a function to grade the idea with some designed algorithms. Users could decide by the score from the website in the situation such as starting own business, doing assignment, etc. This could support the first stage of projects or products.

There is a similar software named Naver APK which searches the key words on Naver shopping. But our project provides a differentiated service. We have different target and offer a wide range of search results.

Therefore, the study aims to relieve the inconvenience at the beginning phase of the project that people commonly experienced.

\section{REQUIREMENTS}
We organized the contents in order of using the website.

\textbf{Steps for using IF (Idea Finder)}
\begin{enumerate}
    \item User J accesses www.ideafinder.com
    \item what J sees first is the search box with the comment “check your idea score!!”
    \item J searches the idea that he wants to verify whether it is new or not
    \item J checks his idea score(ex. When he gets 63 points, comment-boring! Your idea already exists!)
    \item J clicks the ‘details’ button to find the similar ideas on an Intellectual property office website, search engine, App store, Play store and Google scholar.
\end{enumerate}


    \begin{center}
    \caption{\textit{Role Assignments}}
    \begin{tabular}{|c|c|c|}
    \hline
    \multicolumn{2}{|c|}{Group Name} & IFeun-IDLE \\
    \hline
      Roles & Name & Task\\
    \hline
      Development manager & Kyeong-Tae Koo & Back-end developer(API) \\
    \hline
      Software developer, Customer & Oh-Hyeon Kwon & Front-end developer(UX)\\
    \hline
      Software developer, User & Jihye Lee & Back-end developer(server, database)\\
    \hline
      Software developer, User & Juyoun Baek & Front-end developer(UI)\\
    \hline
    \end{tabular}
    \end{center}

\section{DEVELOPMENT ENVIRONMENT}
\begin{itemize}
\item 
\textit{A. Choice of Software Development Platform}

We will conduct web development based on Linux. The reason for choosing web-based environment is it can be executed on any operating system and any search engine. It also allows for a various visual designs and UI within the browser and it is easy to update. Finally, being able to use it without regard to the environment, it can be installed anytime with internet.

When we do web development, at the front-end side, we will use HTML5, CSS, and Javascript. And at the back-end side, we will use PHP, and MySQL for database. If possible, we will also try to implement a cloud-based database environment.

These are the list of selected tools and programming languages for our development.

   \begin{center}
    \begin{tabular}{|c|p{11cm}|}
    \hline
      Tools and language & Reason\\
    \hline
     Java script & This is because JavaScript is indispensable to most Web browsers and is essential for web site development, since it is necessary to adjust the objects in a web page and use various APIs provided by the browser. \\
    \hline
      HTML, CSS & HTML is a style rule language that defines paragraphs, titles, tables, images, movies, etc. on a web page, gives its structure and meaning, and CSS designates HTML content by designating the background color, fonts, content layout and the like. These two are the most used tools and we thought it would be better to use them.\\
    \hline
      MySQL, User & A database provides much more functionality than a file. That is to say, a system specialized for storing information. We need a database that stores a large amount of information. So we wanted to use MySQL, which is open source and free to use.\\
    \hline
      PHP, User & PHP is a server-side programming language that generates HTML programmatically and interacts with the database to store and represent data. PHP is a language for the web that is built for the web and is still evolving for the web, so we think it would be nice to use PHP.\\
    \hline
      Python, User & We will use Python as a crawling method as its modules are helpful to collect web data.\\
    \hline
    \end{tabular}
    \end{center}

\textit{Cost Estimation}

    \begin{tabular}{|c|c|}
    \hline
      Device & Price(won)\\
    \hline
      Domain & 28,600(1year)\\
    \hline
    \end{tabular}
\newline
\newline
We will only use domain cost for the web development as MySQL is free.
\item
\textit{B. Software in Use}
\begin{enumerate}
 \item 
 \textit{Processing Keywords for Search}
 
 It is crucial to understand the meaning of users’ search for better answers. According to Google, to match users’ queries with the information that browser offering, searching algorithms should look up the search terms in the index to find the appropriate pages. Then ranking the best pages can evaluate the web pages by its usability. The algorithm for these processes Google uses is PageRank.
 
 \item 
 \textit{Web Crawling}
 To make a web crawler, we need one module to handle with http request and response, and the other is for parsing HTML. In the former process, Python’s request module that is in urllib module. Then, parser named ‘Beautiful Soup’ would be helpful to create parsing objects.
 
	Also, we will get information for matching and grading by using HTTrack, which is one of the common tools for web scraping. And the data will be extracted from open API. This can be supported by Postman, which is a powerful GUI platform to make API development faster and easier for building API requests through testing, documentation and sharing.

 \item 
 \textit{Grading Method}
 We will use an edited version of Weighted Graph with the nodes of evaluation categories; such as Intellectual property, App Store, Play Store and Google Scholar. We will weight each edge according to the reliability of the information. And each branch needs a Backtracking algorithm to find the matching results. Data collected from API will be saved as a tree structure whose nodes and edges are created from each matching category node, and specific word will have deeper depth. If it needs to search for a deeper node for corresponding result, the grade will get lower. These searching process will repeat on all category nodes.
 \item 
 \textit{Management of Database}
 We will use DBeaver as an editor program for SQL Editor Database management system. Also, Amazon Relational Database Service (RDBMS) of Amazon Web Service (AWS) can be a helpful tool for our project because it is compatible with MySQL.
 
\end{enumerate}
\item
\textit{C. Task Distribution}
\end{itemize}

\section{SPECIFICATIONS}

\begin{itemize}
    \item \textit{A. Website}
    
    Website are made up of two parts that are Front-end and Back-end. We use basic language like Html, CSS, Javascript, MySQL and so on. If we have the skills we need, we'll add them again
    
    \item \textit{B. Website Prototype}
    
    (Picture)
    
    \item \textit{C. Search}
    
    When a user attempts a search, it provides the information they have through database or web crawling. 
    
    \item \textit{D. Grading}
    
    We will score them through our own algorithms.
    
    \item \textit{E. Detail}
    
    Our services offer even more ideas through crawl and API. You will get information about the app store, Google Play Store, search engine, etc.
    
    
\end{itemize}



\end{document}

